% Options for packages loaded elsewhere
\PassOptionsToPackage{unicode}{hyperref}
\PassOptionsToPackage{hyphens}{url}
%
\documentclass[
]{book}
\usepackage{amsmath,amssymb}
\usepackage{lmodern}
\usepackage{iftex}
\ifPDFTeX
  \usepackage[T1]{fontenc}
  \usepackage[utf8]{inputenc}
  \usepackage{textcomp} % provide euro and other symbols
\else % if luatex or xetex
  \usepackage{unicode-math}
  \defaultfontfeatures{Scale=MatchLowercase}
  \defaultfontfeatures[\rmfamily]{Ligatures=TeX,Scale=1}
\fi
% Use upquote if available, for straight quotes in verbatim environments
\IfFileExists{upquote.sty}{\usepackage{upquote}}{}
\IfFileExists{microtype.sty}{% use microtype if available
  \usepackage[]{microtype}
  \UseMicrotypeSet[protrusion]{basicmath} % disable protrusion for tt fonts
}{}
\makeatletter
\@ifundefined{KOMAClassName}{% if non-KOMA class
  \IfFileExists{parskip.sty}{%
    \usepackage{parskip}
  }{% else
    \setlength{\parindent}{0pt}
    \setlength{\parskip}{6pt plus 2pt minus 1pt}}
}{% if KOMA class
  \KOMAoptions{parskip=half}}
\makeatother
\usepackage{xcolor}
\usepackage{longtable,booktabs,array}
\usepackage{calc} % for calculating minipage widths
% Correct order of tables after \paragraph or \subparagraph
\usepackage{etoolbox}
\makeatletter
\patchcmd\longtable{\par}{\if@noskipsec\mbox{}\fi\par}{}{}
\makeatother
% Allow footnotes in longtable head/foot
\IfFileExists{footnotehyper.sty}{\usepackage{footnotehyper}}{\usepackage{footnote}}
\makesavenoteenv{longtable}
\usepackage{graphicx}
\makeatletter
\def\maxwidth{\ifdim\Gin@nat@width>\linewidth\linewidth\else\Gin@nat@width\fi}
\def\maxheight{\ifdim\Gin@nat@height>\textheight\textheight\else\Gin@nat@height\fi}
\makeatother
% Scale images if necessary, so that they will not overflow the page
% margins by default, and it is still possible to overwrite the defaults
% using explicit options in \includegraphics[width, height, ...]{}
\setkeys{Gin}{width=\maxwidth,height=\maxheight,keepaspectratio}
% Set default figure placement to htbp
\makeatletter
\def\fps@figure{htbp}
\makeatother
\setlength{\emergencystretch}{3em} % prevent overfull lines
\providecommand{\tightlist}{%
  \setlength{\itemsep}{0pt}\setlength{\parskip}{0pt}}
\setcounter{secnumdepth}{5}
\usepackage{booktabs}
\usepackage{amsthm}
\makeatletter
\def\thm@space@setup{%
  \thm@preskip=8pt plus 2pt minus 4pt
  \thm@postskip=\thm@preskip
}
\makeatother
\usepackage{booktabs}
\usepackage{longtable}
\usepackage{array}
\usepackage{multirow}
\usepackage{wrapfig}
\usepackage{float}
\usepackage{colortbl}
\usepackage{pdflscape}
\usepackage{tabu}
\usepackage{threeparttable}
\usepackage{threeparttablex}
\usepackage[normalem]{ulem}
\usepackage{makecell}
\usepackage{xcolor}
\ifLuaTeX
  \usepackage{selnolig}  % disable illegal ligatures
\fi
\usepackage[]{natbib}
\bibliographystyle{apalike}
\IfFileExists{bookmark.sty}{\usepackage{bookmark}}{\usepackage{hyperref}}
\IfFileExists{xurl.sty}{\usepackage{xurl}}{} % add URL line breaks if available
\urlstyle{same} % disable monospaced font for URLs
\hypersetup{
  pdftitle={BID: Pobreza Energética},
  pdfauthor={Centro de Inteligencia Territorial - UAI},
  hidelinks,
  pdfcreator={LaTeX via pandoc}}

\title{BID: Pobreza Energética}
\author{Centro de Inteligencia Territorial - UAI}
\date{2022-08-05}

\begin{document}
\maketitle

{
\setcounter{tocdepth}{1}
\tableofcontents
}
\hypertarget{introducciuxf3n}{%
\chapter{Introducción}\label{introducciuxf3n}}

Apoyo para una Transición Energética Justa, Limpia y Sostenible en Chile:

\textbf{Estudio para la medición de indicadores de pobreza energética}

\hypertarget{objetivos}{%
\chapter{Objetivos}\label{objetivos}}

\hypertarget{objetivos-generales}{%
\section{Objetivos Generales}\label{objetivos-generales}}

Medir Indicadores de Probreza Energética en el sector residencial, con la escala más detallada posible en base a la información ya disponible en otros estudios y bases de datos, con la que actualmente cuente el Misterio, o que pueda gestionar de manera efectiva para su utilización posterior.

\hypertarget{objetivos-especuxedficos}{%
\section{Objetivos Específicos}\label{objetivos-especuxedficos}}

\begin{itemize}
\tightlist
\item
  Desarrollar y definir metodologías para la contrucción de indicadores.
\end{itemize}

\hypertarget{m_teorico}{%
\chapter{Marco Teórico}\label{m_teorico}}

\textbf{¿Qué es pobreza Energética?}

Nos referimos a Pobreza Energética cuando un hogar no tiene acceso equitativo a servicios energéticos de alta calidad (es decir, que sean adecuados, confiables, no contaminantes y seguros) para cubrir sus necesidades fundamentales y básicas que permitan sostener el desarrollo humano y económico de sus integrantes. \emph{Fuente:\url{http://redesvid.uchile.cl/pobreza-energetica/panel-de-indicadores/}}

Un hogar se encuentra en situación de pobreza energética cuando no tiene acceso equitativo a servicios energéticos de alta calidad para cubrir sus necesidades fundamentales y básicas, que permitan sostener el desarrollo humano y económico de sus miembros. Las necesidades fundamentales son aquellas que implican impactos directos en la salud humana; mientras que las necesidades básicas corresponden a aquellos requerimientos energéticos cuya pertinencia depende de las particularidades culturales y territoriales.

\begin{figure}
\centering
\includegraphics{images/esquema_PE.png}
\caption{Esquema conceptual Pobreza Energética - \href{http://redesvid.uchile.cl/pobreza-energetica/wp-content/uploads/2020/09/Policy-Paper-Pobreza-Energe\%CC\%81tica.-El-acceso-desigual-a-energi\%CC\%81a-de-calidad-como-barrera-para-el-desarrollo-en-Chile.pdf}{Universidad de Chile}}
\end{figure}

La pobreza energética se expresa en relación a dos grupos de necesidades: `\emph{fundamentales}' y `\emph{básicas}'. Las necesidades \textbf{fundamentales} son aquellas que implican impactos directos en la salud humana. Su satisfacción se considera crítica, independiente del contexto territorial. La cocción y conservación de alimentos, las temperaturas mínima y máxima saludables, el acceso al agua y la disponibilidad de suministro eléctrico continuo para personas electrodependientes en salud, se encuentran en este primer grupo. Por otra parte, las necesidades \textbf{básicas} corresponden a aquellos requerimientos cuya pertinencia depende de las características socioecológicas (biofísicas, geográficas y climáticas), sociotécnicas (tecnológicas e infraestructurales) y socioculturales (normas, mercados, costumbres y expectativas relacionadas con calidad de vida y desarrollo humano), propias de un determinado territorio. El confort térmico, el agua caliente sanitaria, la iluminación, los electrodomésticos y dispositivos tecnológicos para la educación son ejemplos de este segundo grupo.

Mientras las necesidades fundamentales se consideran de forma universal, las necesidades básicas requieren de una definición y ponderación en función de su pertinencia para una población en particular, situada en un territorio, en un contexto temporal definido y bajo condiciones socioculturales específicas.

\begin{figure}
\centering
\includegraphics{images/necedidades_energia.png}
\caption{Necesidades fundamentales y básica de energía - \href{http://redesvid.uchile.cl/pobreza-energetica/wp-content/uploads/2020/09/Policy-Paper-Pobreza-Energe\%CC\%81tica.-El-acceso-desigual-a-energi\%CC\%81a-de-calidad-como-barrera-para-el-desarrollo-en-Chile.pdf}{Universidad de Chile}}
\end{figure}

\textbf{Referencias:}

\href{http://redesvid.uchile.cl/pobreza-energetica/wp-content/uploads/2020/09/Policy-Paper-Pobreza-Energe\%CC\%81tica.-El-acceso-desigual-a-energi\%CC\%81a-de-calidad-como-barrera-para-el-desarrollo-en-Chile.pdf}{Pobreza Energética- El Acceso desigual a energía de caldiad como barrera para el desarrollo en Chile. Univerisad de Chile, 2020}

\hypertarget{data_explorer}{%
\chapter{Exploración de Datos}\label{data_explorer}}

Las bases de datos recibidas de Probreza Energética (27-07-2022), corresponden a 3 carpetas, las cuales en el presente capítulo se procederá un análisis descriptivo general.

\begin{figure}
\centering
\includegraphics{images/bbdd_folders.png}
\caption{Estructura de carpeta de Bases de Datos}
\end{figure}

\hypertarget{bbdd-mapa-vulneravilidad}{%
\section{\texorpdfstring{\texttt{BBDD\ mapa\ vulneravilidad}}{BBDD mapa vulneravilidad}}\label{bbdd-mapa-vulneravilidad}}

\hypertarget{descripciuxf3n-de-variables}{%
\subsection{Descripción de Variables}\label{descripciuxf3n-de-variables}}

\textbf{Nombre Archivo}: ``\emph{Detalle de campos.xlsx}''

\begin{table}
\centering
\begin{tabular}{c|c|c}
\hline
Nombre campo & Descripción & Fuente\\
\hline
REGION & Código región & INE, Censo 2017\\
\hline
PROVINCIA & Código provincia & INE, Censo 2017\\
\hline
COMUNA & Código comuna & INE, Censo 2017\\
\hline
NOM\_REGION & Nombre región & INE, Censo 2017\\
\hline
NOM\_PROVIN & Nombre provincia & INE, Censo 2017\\
\hline
NOM\_COMUNA & Nombre comuna & INE, Censo 2017\\
\hline
DESTINO\_VI & Uso o destino de la vivienda 
2: vivienda colectiva
3: vivienda de temporada
4: vivienda desocupada
5: vivienda ocupada con moradores ausentes
6: vivienda ocupada con moradores presentes & INE, Pre censo 2016\\
\hline
NOM\_DIRECC & Nombre de calle o camino de referencia & INE, Pre censo 2016\\
\hline
N\_LETRA & Numeración & INE, Pre censo 2016\\
\hline
LOCALIDAD & Nombre de localidad, área rural & INE, Censo 2017\\
\hline
ENTIDAD & Nombre de entidad, área rural & INE, Censo 2017\\
\hline
CATEGORIA & Nombres de categorías de asentamiento humano en área rural & INE, Censo 2017\\
\hline
EMPRESA\_ID & Identificador empresa distribuidora & SEC, 2018\\
\hline
EMPRESA\_1 & Nombre empresa distribuidora & SEC, 2018\\
\hline
SSAA & 1: vivienda se ubica en sistema aislado
0: vivienda no se ubica en sistema aislado & Catastro SSAA, 2018\\
\hline
NOM\_SSAA & Nombre del sistema aislado & Catastro SSAA, 2018\\
\hline
TIPO\_SUM\_S & Tipo de suministro del sistema eléctrico aislado: 
Parcial: menos de 24 horas al día
Permanente: 24 horas al día & Catastro SSAA, 2018\\
\hline
FV\_IND & 1: vivienda tiene sistema individual de autogeneración
0: vivienda no tiene sistema individual de autogeneración & Catastro sistemas individuales, 2018\\
\hline
NOM\_FV\_IND & Nombre del proyecto que dio origen al sistema individual & Catastro sistemas individuales, 2018\\
\hline
SUMINS\_FV & Tipo de suministro del sistema individual: 
Parcial: menos de 24 horas al día
Permanente: 24 horas al día & Catastro sistemas individuales, 2018\\
\hline
PROY\_ELECT & 1: vivienda está incluida en algún proyecto de electrificación
0: vivienda no está incluida en algún proyecto de electrificación & Varias fuentes\\
\hline
EST\_PROY\_E & Estado del proyecto: ejecutado, en ejecución, RS, con financiamiento, en licitación. & Varias fuentes\\
\hline
NOM\_PRO\_EL & Nombre del proyecto & Varias fuentes\\
\hline
CODIGO\_IDI & Código IDI del proyecto (código del banco integrado de proyectos) & Varias fuentes\\
\hline
X & Longitud & Obtenida en Arcgis\\
\hline
Y & Latitud & Obtenida en Arcgis\\
\hline
\end{tabular}
\end{table}

\hypertarget{espaciales}{%
\subsection{Espaciales}\label{espaciales}}

\hypertarget{data_clean}{%
\chapter{Limpieza de Datos}\label{data_clean}}

  \bibliography{book.bib,packages.bib}

\end{document}
