% Options for packages loaded elsewhere
\PassOptionsToPackage{unicode}{hyperref}
\PassOptionsToPackage{hyphens}{url}
%
\documentclass[
]{book}
\usepackage{amsmath,amssymb}
\usepackage{lmodern}
\usepackage{iftex}
\ifPDFTeX
  \usepackage[T1]{fontenc}
  \usepackage[utf8]{inputenc}
  \usepackage{textcomp} % provide euro and other symbols
\else % if luatex or xetex
  \usepackage{unicode-math}
  \defaultfontfeatures{Scale=MatchLowercase}
  \defaultfontfeatures[\rmfamily]{Ligatures=TeX,Scale=1}
\fi
% Use upquote if available, for straight quotes in verbatim environments
\IfFileExists{upquote.sty}{\usepackage{upquote}}{}
\IfFileExists{microtype.sty}{% use microtype if available
  \usepackage[]{microtype}
  \UseMicrotypeSet[protrusion]{basicmath} % disable protrusion for tt fonts
}{}
\makeatletter
\@ifundefined{KOMAClassName}{% if non-KOMA class
  \IfFileExists{parskip.sty}{%
    \usepackage{parskip}
  }{% else
    \setlength{\parindent}{0pt}
    \setlength{\parskip}{6pt plus 2pt minus 1pt}}
}{% if KOMA class
  \KOMAoptions{parskip=half}}
\makeatother
\usepackage{xcolor}
\usepackage{longtable,booktabs,array}
\usepackage{calc} % for calculating minipage widths
% Correct order of tables after \paragraph or \subparagraph
\usepackage{etoolbox}
\makeatletter
\patchcmd\longtable{\par}{\if@noskipsec\mbox{}\fi\par}{}{}
\makeatother
% Allow footnotes in longtable head/foot
\IfFileExists{footnotehyper.sty}{\usepackage{footnotehyper}}{\usepackage{footnote}}
\makesavenoteenv{longtable}
\usepackage{graphicx}
\makeatletter
\def\maxwidth{\ifdim\Gin@nat@width>\linewidth\linewidth\else\Gin@nat@width\fi}
\def\maxheight{\ifdim\Gin@nat@height>\textheight\textheight\else\Gin@nat@height\fi}
\makeatother
% Scale images if necessary, so that they will not overflow the page
% margins by default, and it is still possible to overwrite the defaults
% using explicit options in \includegraphics[width, height, ...]{}
\setkeys{Gin}{width=\maxwidth,height=\maxheight,keepaspectratio}
% Set default figure placement to htbp
\makeatletter
\def\fps@figure{htbp}
\makeatother
\setlength{\emergencystretch}{3em} % prevent overfull lines
\providecommand{\tightlist}{%
  \setlength{\itemsep}{0pt}\setlength{\parskip}{0pt}}
\setcounter{secnumdepth}{5}
\usepackage{booktabs}
\usepackage{amsthm}
\makeatletter
\def\thm@space@setup{%
  \thm@preskip=8pt plus 2pt minus 4pt
  \thm@postskip=\thm@preskip
}
\makeatother
\ifLuaTeX
  \usepackage{selnolig}  % disable illegal ligatures
\fi
\usepackage[]{natbib}
\bibliographystyle{apalike}
\IfFileExists{bookmark.sty}{\usepackage{bookmark}}{\usepackage{hyperref}}
\IfFileExists{xurl.sty}{\usepackage{xurl}}{} % add URL line breaks if available
\urlstyle{same} % disable monospaced font for URLs
\hypersetup{
  pdftitle={BID: Probreza Energética},
  pdfauthor={Centro de Inteligencia Territorial - UAI},
  hidelinks,
  pdfcreator={LaTeX via pandoc}}

\title{BID: Probreza Energética}
\author{Centro de Inteligencia Territorial - UAI}
\date{2022-08-05}

\begin{document}
\maketitle

{
\setcounter{tocdepth}{1}
\tableofcontents
}
\hypertarget{introducciuxf3n}{%
\chapter{Introducción}\label{introducciuxf3n}}

Apoyo para una Transición Energética Justa, Limpia y Sostenible en Chile:

\textbf{Estudio para la medición de indicadores de pobreza energética}

Por completar\ldots.

\hypertarget{data_clean}{%
\chapter{Limpieza de Datos}\label{data_clean}}

\hypertarget{data_explorer}{%
\chapter{Exploración de Datos}\label{data_explorer}}

\hypertarget{lectura-de-archivos}{%
\section{Lectura de Archivos}\label{lectura-de-archivos}}

\hypertarget{espaciales}{%
\subsection{Espaciales}\label{espaciales}}

\hypertarget{m_teorico}{%
\chapter{Marco Teórico}\label{m_teorico}}

\textbf{¿Qué es pobreza Energética?}

Nos referimos a Pobreza Energética cuando un hogar no tiene acceso equitativo a servicios energéticos de alta calidad (es decir, que sean adecuados, confiables, no contaminantes y seguros) para cubrir sus necesidades fundamentales y básicas que permitan sostener el desarrollo humano y económico de sus integrantes. \emph{Fuente:\url{http://redesvid.uchile.cl/pobreza-energetica/panel-de-indicadores/}}

\hypertarget{objetivos}{%
\chapter{Objetivos}\label{objetivos}}

\hypertarget{objetivos-generales}{%
\section{Objetivos Generales}\label{objetivos-generales}}

Medir Indicadores de Probreza Energética en el sector residencial, con la escala más detallada posible en base a la información ya disponible en otros estudios y bases de datos, con la que actualmente cuente el Misterio, o que pueda gestionar de manera efectiva para su utilización posterior.

\hypertarget{objetivos-especuxedficos}{%
\section{Objetivos Específicos}\label{objetivos-especuxedficos}}

\begin{itemize}
\tightlist
\item
  Desarrollar y definir metodologías para la contrucción de indicadores.
\end{itemize}

  \bibliography{book.bib,packages.bib}

\end{document}
